\documentclass[11pt]{article}
\usepackage{trishan1}
\usepackage{lipsum}  % for sample text

\titles{Title}
\author{\textsc{Author name} \\[0.15cm]
  \href{https://www.isibang.ac.in}{Indian Statistical Institute, Bangalore}}
\date{}


\begin{document}
  \maketitle
 \abstract{\lipsum[0-1] }
\section{section 1}
 
\subsection{subsection} 
      \lipsum[1-1] $\extt, \hmm{R}, \tens{A}{B},$ (this was a test) \les \es \ess \Qed

      \begin{Thm}{cshh}{}
             \lipsum[1-1]
      \end{Thm}

     \lipsum[1-1]

      \coro{what is that} \lipsum[2-2]
     
    \noindent What to do now ? let's see how can we write lemma. 

    \Lem{\lipsum[4-4]}

    \noindent \lipsum[5-5]

    \begin{Def}{(Fundamental Class) }{}
          \lipsum[5-5]
    \end{Def}
    
\end{document}